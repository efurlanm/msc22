%%%%%%%%%%%%%%%%%%%%%%%%%%%%%%%%%%%%%%%%%%%%%%%%%%%%%%
%Contracapa
%%%%%%%%%%%%%%%%%%%%%%%%%%%%%%%%%%%%%%%%%%%%%%%%%%%%%%

\thispagestyle{empty}
 \begin{table}
  \begin{center}
   \begin{tabularx}{\textwidth}{X}
   \textbf{PUBLICAÇÕES TÉCNICO-CIENTÍFICAS EDITADAS PELO INPE}
  \end{tabularx} 
  \end{center}
 \end{table}
  
 \begin{table}
\begin{center}
%\begin{tabularx}{\textwidth}{X X}
\begin{tabularx}{\textwidth}{>{\raggedright\arraybackslash}X >{\raggedright\arraybackslash}X}      
  \textbf{Teses e Dissertações (TDI)}              & \textbf{Manuais Técnicos (MAN)}\\
\\
Teses e Dissertações apresentadas nos Cursos de Pós-Graduação do INPE.	&
São publicações de caráter técnico que incluem normas, procedimentos, instruções e orientações.\\
\\
\textbf{Notas Técnico-Científicas (NTC)}           & \textbf{Relatórios de Pesquisa (RPQ)}\\
\\
Incluem resultados preliminares de pesquisa, descrição de equipamentos, descrição e ou documentação de programas de computador, descrição de sistemas e experimentos, apresentação de testes, dados, atlas, e documentação de projetos de engenharia. 
&	
Reportam resultados ou progressos de pesquisas tanto de natureza técnica quanto científica, cujo nível seja compatível com o de uma publicação em periódico nacional ou internacional.\\
\\
\textbf{Propostas e Relatórios de Projetos (PRP)}	& \textbf{Publicações Didáticas (PUD)} 
\\
\\
São propostas de projetos técnico-científicos e relatórios de acompanhamento de projetos, atividades e convênios.
&	
Incluem apostilas, notas de aula e manuais didáticos. \\
\\         
\textbf{Publicações Seriadas} 	& \textbf{Programas de Computador (PDC)}\\
\\
São os seriados técnico-científicos: boletins, periódicos, anuários e anais de eventos (simpósios e congressos). Constam destas publicações o Internacional Standard Serial Number (ISSN), que é um código único e definitivo para identificação de títulos de seriados. 
&	
São a seqüência de instruções ou códigos, expressos em uma linguagem de programação compilada ou interpretada, a ser executada por um computador para alcançar um determinado objetivo. Aceitam-se tanto programas fonte quanto os executáveis.\\
\\
\textbf{Pré-publicações (PRE)} \\
\\
Todos os artigos publicados em  periódicos, anais e como capítulos de livros.    \end{tabularx}
\end{center}
\end{table}

