%     _                               _    
%    / \   _ __  _ __   _____  __    / \   
%   / _ \ | '_ \| '_ \ / _ \ \/ /   / _ \  
%  / ___ \| | | | | | |  __/>  <   / ___ \ 
% /_/   \_\_| |_|_| |_|\___/_/\_\ /_/   \_\
%%%%%%%%%%%%%%%%%%%%%%%%%%%%%%%%%%%%%%%%%%%%%%%%%%%%%%%
\renewcommand{\thechapter}{A}
\chapter{ANNEX A - STENCIL CODE}
\label{annexA}
%%%%%%%%%%%%%%%%%%%%%%%%%%%%%%%%%%%%%%%%%%%%%%%%%%%%%%%%

This annex shows the original codes developed in C by Torsten Hoefler \cite {Balaji2017}, which were used as a reference to write the serial and parallel implementations of the stencil case study used in this work.

\section{Serial version}
\label{ann_bljser}

Original untouched serial version. Uses two arrays to store the grid, one holds data during execution and the other holds the result of calculations. Two loops are used to iterate through the 2D matrix and update grid points. Three fixed array points are used to insert heat units into each repeating loop. A variable is used to accumulate the amount of heat inserted in each loop (note: in the implementation used in this work, the variable was moved out of the main loop. See \autoref{sec_stenimplf90}).

\begin{lstlisting}[language=C, label={lst_stencseri}, caption={stencil.c.}]
#include <stdio.h>
#include <stdlib.h>
#include <math.h>
#include <mpi.h>

// row-major order
#define ind(i,j) (j)*n+i

void printarr(double *a, int n) {
// does nothing right now, should record each "frame" as image
FILE *fp = fopen("heat.svg", "w");
const int size = 5;

fprintf(fp, "<html>\n<body>\n<svg xmlns=\"http://www.w3.org/2000/svg\" version=\"1.1\">");

fprintf(fp, "\n<rect x=\"0\" y=\"0\" width=\"%i\" height=\"%i\" style=\"stroke-width:1;fill:rgb(0,0,0);stroke:rgb(0,0,0)\"/>", size*n, size*n);
for(int i=1; i<n+1; ++i)
for(int j=1; j<n+1; ++j) {
int rgb = (a[ind(i,j)] > 0) ? rgb = (int)round(255.0*a[ind(i,j)]) : 0.0;
if(rgb>255) rgb=255;
if(rgb) fprintf(fp, "\n<rect x=\"%i\" y=\"%i\" width=\"%i\" height=\"%i\" style=\"stroke-width:1;fill:rgb(%i,0,0);stroke:rgb(%i,0,0)\"/>", size*(i-1), size*(j-1), size, size, rgb, rgb);
}
fprintf(fp, "</svg>\n</body>\n</html>");

fclose(fp);
}

int main(int argc, char **argv) {

int n = atoi(argv[1]); // nxn grid
int energy = atoi(argv[2]); // energy to be injected per iteration
int niters = atoi(argv[3]); // number of iterations
double *aold = (double*)calloc(1,(n+2)*(n+2)*sizeof(double)); // 1-wide halo zones!
double *anew = (double*)calloc(1,(n+2)*(n+2)*sizeof(double)); // 1-wide halo-zones!
double *tmp;

MPI_Init(NULL, NULL);

#define nsources 3
int sources[nsources][2] = {{n/2, n/2}, {n/3, n/3}, {n*4/5, n*8/9}};

double heat=0.0; // total heat in system
double t=-MPI_Wtime();
for(int iter=0; iter<niters; ++iter) {
for(int j=1; j<n+1; ++j) {
for(int i=1; i<n+1; ++i) {
anew[ind(i,j)] = aold[ind(i,j)]/2.0 + (aold[ind(i-1,j)] + aold[ind(i+1,j)] + aold[ind(i,j-1)] + aold[ind(i,j+1)])/4.0/2.0;
heat += anew[ind(i,j)];
}
}
for(int i=0; i<nsources; ++i) {
anew[ind(sources[i][0],sources[i][1])] += energy; // heat source
}
tmp=anew; anew=aold; aold=tmp; // swap arrays
}
t+=MPI_Wtime();
printarr(anew, n);
printf("last heat: %f time: %f\n", heat, t);

MPI_Finalize();
}
\end{lstlisting}





%----------------------------------------
\section{Parallel version}
\label{bljCpar}

Original untouched code from the parallel version. Divides the grid into parts, and each part is calculated by an MPI process. Communication between the processes is necessary, because to calculate an edge point, it is necessary to know the value of the point that is in the adjacent process.

\begin{lstlisting}[language=C, label={lst_stencpara}, caption={stencil\_mpi.c.}]
/*
* Copyright (c) 2012 Torsten Hoefler. All rights reserved.
*
* Author(s): Torsten Hoefler <htor@illinois.edu>
*
*/

#include "stencil_par.h"

int main(int argc, char **argv) {

MPI_Init(&argc, &argv);
int r,p;
MPI_Comm comm = MPI_COMM_WORLD;
MPI_Comm_rank(comm, &r);
MPI_Comm_size(comm, &p);
int n, energy, niters, px, py;

if (r==0) {
// argument checking
if(argc < 6) {
if(!r) printf("usage: stencil_mpi <n> <energy> <niters> <px> <py>\n");
MPI_Finalize();
exit(1);
}

n = atoi(argv[1]); // nxn grid
energy = atoi(argv[2]); // energy to be injected per iteration
niters = atoi(argv[3]); // number of iterations
px=atoi(argv[4]); // 1st dim processes
py=atoi(argv[5]); // 2nd dim processes

if(px * py != p) MPI_Abort(comm, 1);// abort if px or py are wrong
if(n % py != 0) MPI_Abort(comm, 2); // abort px needs to divide n
if(n % px != 0) MPI_Abort(comm, 3); // abort py needs to divide n

// distribute arguments
int args[5] = {n, energy, niters, px,  py};
MPI_Bcast(args, 5, MPI_INT, 0, comm);
}
else {
int args[5];
MPI_Bcast(args, 5, MPI_INT, 0, comm);
n=args[0]; energy=args[1]; niters=args[2]; px=args[3]; py=args[4];
}

// determine my coordinates (x,y) -- r=x*a+y in the 2d processor array
int rx = r % px;
int ry = r / px;
// determine my four neighbors
int north = (ry-1)*px+rx; if(ry-1 < 0)   north = MPI_PROC_NULL;
int south = (ry+1)*px+rx; if(ry+1 >= py) south = MPI_PROC_NULL;
int west= ry*px+rx-1;     if(rx-1 < 0)   west = MPI_PROC_NULL;
int east = ry*px+rx+1;    if(rx+1 >= px) east = MPI_PROC_NULL;
// decompose the domain
int bx = n/px; // block size in x
int by = n/py; // block size in y
int offx = rx*bx; // offset in x
int offy = ry*by; // offset in y

//printf("%i (%i,%i) - w: %i, e: %i, n: %i, s: %i\n", r, ry,rx,west,east,north,south);

// allocate two work arrays
double *aold = (double*)calloc(1,(bx+2)*(by+2)*sizeof(double)); // 1-wide halo zones!
double *anew = (double*)calloc(1,(bx+2)*(by+2)*sizeof(double)); // 1-wide halo zones!
double *tmp;

// initialize three heat sources
#define nsources 3
int sources[nsources][2] = {{n/2,n/2}, {n/3,n/3}, {n*4/5,n*8/9}};
int locnsources=0; // number of sources in my area
int locsources[nsources][2]; // sources local to my rank
for (int i=0; i<nsources; ++i) { // determine which sources are in my patch
int locx = sources[i][0] - offx;
int locy = sources[i][1] - offy;
if(locx >= 0 && locx < bx && locy >= 0 && locy < by) {
locsources[locnsources][0] = locx+1; // offset by halo zone
locsources[locnsources][1] = locy+1; // offset by halo zone
locnsources++;
}
}

double t=-MPI_Wtime(); // take time
// allocate communication buffers
double *sbufnorth = (double*)calloc(1,bx*sizeof(double)); // send buffers
double *sbufsouth = (double*)calloc(1,bx*sizeof(double));
double *sbufeast = (double*)calloc(1,by*sizeof(double));
double *sbufwest = (double*)calloc(1,by*sizeof(double));
double *rbufnorth = (double*)calloc(1,bx*sizeof(double)); // receive buffers
double *rbufsouth = (double*)calloc(1,bx*sizeof(double));
double *rbufeast = (double*)calloc(1,by*sizeof(double));
double *rbufwest = (double*)calloc(1,by*sizeof(double));

double heat; // total heat in system
for(int iter=0; iter<niters; ++iter) {
// refresh heat sources
for(int i=0; i<locnsources; ++i) {
aold[ind(locsources[i][0],locsources[i][1])] += energy; // heat source
}

// exchange data with neighbors
MPI_Request reqs[8];
for(int i=0; i<bx; ++i) sbufnorth[i] = aold[ind(i+1,1)]; // pack loop - last valid region
for(int i=0; i<bx; ++i) sbufsouth[i] = aold[ind(i+1,by)]; // pack loop
for(int i=0; i<by; ++i) sbufeast[i] = aold[ind(bx,i+1)]; // pack loop
for(int i=0; i<by; ++i) sbufwest[i] = aold[ind(1,i+1)]; // pack loop
MPI_Isend(sbufnorth, bx, MPI_DOUBLE, north, 9, comm, &reqs[0]);
MPI_Isend(sbufsouth, bx, MPI_DOUBLE, south, 9, comm, &reqs[1]);
MPI_Isend(sbufeast, by, MPI_DOUBLE, east, 9, comm, &reqs[2]);
MPI_Isend(sbufwest, by, MPI_DOUBLE, west, 9, comm, &reqs[3]);
MPI_Irecv(rbufnorth, bx, MPI_DOUBLE, north, 9, comm, &reqs[4]);
MPI_Irecv(rbufsouth, bx, MPI_DOUBLE, south, 9, comm, &reqs[5]);
MPI_Irecv(rbufeast, by, MPI_DOUBLE, east, 9, comm, &reqs[6]);
MPI_Irecv(rbufwest, by, MPI_DOUBLE, west, 9, comm, &reqs[7]);
MPI_Waitall(8, reqs, MPI_STATUSES_IGNORE);
for(int i=0; i<bx; ++i) aold[ind(i+1,0)] = rbufnorth[i]; // unpack loop - into ghost cells
for(int i=0; i<bx; ++i) aold[ind(i+1,by+1)] = rbufsouth[i]; // unpack loop
for(int i=0; i<by; ++i) aold[ind(bx+1,i+1)] = rbufeast[i]; // unpack loop
for(int i=0; i<by; ++i) aold[ind(0,i+1)] = rbufwest[i]; // unpack loop

// update grid points
heat = 0.0;
for(int j=1; j<by+1; ++j) {
for(int i=1; i<bx+1; ++i) {
anew[ind(i,j)] = aold[ind(i,j)]/2.0 + (aold[ind(i-1,j)] + aold[ind(i+1,j)] + aold[ind(i,j-1)] + aold[ind(i,j+1)])/4.0/2.0;
heat += anew[ind(i,j)];
}
}

// swap arrays
tmp=anew; anew=aold; aold=tmp;

// optional - print image
if(iter == niters-1) printarr_par(iter, anew, n, px, py, rx, ry, bx, by, offx, offy, comm);
}
t+=MPI_Wtime();

// get final heat in the system
double rheat;
MPI_Allreduce(&heat, &rheat, 1, MPI_DOUBLE, MPI_SUM, comm);
if(!r) printf("[%i] last heat: %f time: %f\n", r, rheat, t);

MPI_Finalize();
}
\end{lstlisting}


\begin{lstlisting}[language=C, label={lst_stencparainc}, caption={stencil\_par.h.}]
/*
* stencil_par.h
*
*  Created on: Jan 4, 2012
*      Author: htor
*/

#ifndef STENCIL_PAR_H_
#define STENCIL_PAR_H_

#include "mpi.h"
#include <math.h>
#include <stdio.h>
#include <stdlib.h>
#include <string.h>
#include <stdint.h>

// row-major order
#define ind(i,j) (j)*(bx+2)+(i)

void printarr_par(int iter, double* array, int size, int px, int py, int rx, int ry, int bx, int by, int offx, int offy, MPI_Comm comm);

#endif /* STENCIL_PAR_H_ */

\end{lstlisting}

