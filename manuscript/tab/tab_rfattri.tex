\begin{tabular}{m{\dimexpr 0.2\linewidth-2.1\tabcolsep}m{\dimexpr 0.8\linewidth-2.1\tabcolsep}}\toprule			
\textbf{Attribute}	&	\textbf{Description}	\\
\midrule[0.1pt]\vspace{-10pt}			
neo	&	Near-Earth object	\\
pha	&	Potential hazards asteroid	\\
h	&	Absolute magnitude parameter	\\
diameter	&	Asteroid diameter, from equivalent sphere	\\
albedo	&	Geometric albedo	\\
diameter\_sigma	&	Diameter 1 sigma	\\
orbit\_id	&	Orbit id	\\
epoch	&	Epoch, particular time	\\
epoch\_mjd	&	Epoch of the elements represented as the Modified Julian Date (MJD)	\\
epoch\_cal	&	Epoch calender	\\
e	&	The eccentricity of the conic	\\
a	&	Mean Distance, the semi-major axis of the orbit measured in AU	\\
q	&	Perihelion distance [AU]	\\
i	&	Inclination, the angle between the ecliptic plane and the plane of the orbit	\\
om	&	Longitude of the Ascending Node (Omega)	\\
w	&	Argument of Perihelion (w)	\\
ma	&	Mean Anomaly (M)	\\
ad	&	Aphelion distance [AU]	\\
n	&	Mean motion [deg/d]	\\
tp	&	Time of perihelion passage (TDB)	\\
tp\_cal	&	tp calender	\\
per	&	Period	\\
per\_y	&	Period year	\\
moid	&	Earth minimum orbit intersection distance au unit	\\
moid\_ld	&	Earth minimum orbit intersection distance lunar unit	\\
sigma\_e	&	e 1-sigma (see "e" above)	\\
sigma\_a	&	a 1-sigma	\\
sigma\_q	&	q 1-sigma	\\
sigma\_i	&	i 1-sigma	\\
sigma\_om	&	om 1-sigma	\\
sigma\_w	&	w 1-sigma	\\
sigma\_ma	&	ma 1-sigma	\\
sigma\_ad	&	ad 1-sigma	\\
sigma\_n	&	n 1-sigma	\\
sigma\_tp	&	tp 1-sigma	\\
sigma\_per	&	period 1-sigma	\\
rms	&	A measure of the predicted data’s deviation from the observed data	\\
class	&	Asteroid Orbit Classes	\\
\bottomrule\end{tabular}			