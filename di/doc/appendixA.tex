%     _                               _ _           _    
%    / \   _ __  _ __   ___ _ __   __| (_)_  __    / \   
%   / _ \ | '_ \| '_ \ / _ \ '_ \ / _` | \ \/ /   / _ \  
%  / ___ \| |_) | |_) |  __/ | | | (_| | |>  <   / ___ \ 
% /_/   \_\ .__/| .__/ \___|_| |_|\__,_|_/_/\_\ /_/   \_\
%         |_|   |_|  
%%%%%%%%%%%%%%%%%%%%%%%%%%%%%%%%%%%%%%%%%%%%%%%%%%%%%%
\renewcommand{\thechapter}{A}
\chapter{APPENDIX A - PUBLISHED ARTICLES}
\label{appendixA}
%%%%%%%%%%%%%%%%%%%%%%%%%%%%%%%%%%%%%%%%%%%%%%%%%%%%%%

Two articles were submitted and approved for publication, one in a national congress and the other in a journal, and both cover only the five-point stencil case test:

\begin{itemize}

\item MIRANDA, E. F.; STEPHANY, S. \emph{Comparison of High Performance Computing Approaches in the Python Environment for a Five-Point Stencil Test Problem}. In proceedings: XV Brazilian e-Science Workshop (BreSci), 2021. p. 33-40. ISSN 2763-8774. DOI 10.5753/bresci.2021.15786.

Article approved and presented at the national congress (article 213763), XV Brazilian e-Science Workshop (BreSci), event that is part of the XLI Congress of the Brazilian Computer Society (CSBC-2021), from 18 to 23 July 2021. Available from: http://doi.org/10.5753/bresci.2021.15786. Access in: Dec. 2021.

\textbf{Abstract}: Several of the most important high-performance computing approaches available in the Python programming environment of the LNCC Santos Dumont supercomputer, are compared using a specific test problem. Python includes specific libraries, implementations, development tools, documentation, optimization and parallelization resources. It provides a straightforward way to program using a high level of abstraction, but the parallelization features for exploring multiple cores, processors, or accelerators such as GPUs, are diverse and may not be easily chosen by the user. Serial and parallel implementations of a test problem in Fortran 90 are taken as benchmarks to compare performance. This work is a primer for the use of HPC resources in Python.

\item MIRANDA, E. F.; STEPHANY, S. \emph{Common approaches to HPC in Python evaluated for a scientific computing test case}. REVISTA CEREUS, 13(2), 84-98, 2021. ISSN 2175-7275. DOI 10.18605/2175-7275/cereus.v13n2p84-98.

Article accepted in a journal (submission 3408), REVISTA CEREUS. Qualis CAPES Interdisciplinary B2 in the evaluation of the 2013-2016 quadrennium. Available from:  http://doi.org/10.18605/2175-7275/cereus.v13n2p84-98. Access in: Dec. 2021.

\textbf{Abstract}: A number of the most common high-performance computing approaches available in the Python programming environment of the LNCC Santos Dumont supercomputer, are compared using a specific test case. Python includes specific libraries, development tools, implementations, documentation and optimizing/parallelizing resources. It provides a straightforward way to program in a high level of abstraction, but parallelization resources to exploit multiple cores, processors or accelerators like GPUs are diverse and may be not easily selectable by the programmer. This work makes a comparison of common approaches in Python to boost computing performance. The test case is a well-known 2D heat transmission problem modeled by the Poisson partial-differential equation, which is solved by a finite difference method that requires the calculation of a 5-point stencil over the domain grid. Their serial and parallel implementations in Fortran 90 were taken as references in order to compare their performance to some serial and parallel Python implementations of the same algorithm. Besides performance results, a discussion about the trade-off between easiness of programming versus processing performance is included. This work is a primer for the use of HPC resources in Python. 

\end{itemize}

\

\textbf{ORCID}

\begin{itemize}

\item Miranda, E. F.: ORCID 0000-0003-1200-794X

\item Stephany, S.: ORCID 0000-0002-6302-4259

\end{itemize}

